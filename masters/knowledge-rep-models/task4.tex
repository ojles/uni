\documentclass[a4paper,14pt]{extarticle} % <<< use extarticle for 14pt size

\usepackage[T2A]{fontenc}
\usepackage[utf8]{inputenc}
\usepackage[ukrainian]{babel}
\usepackage{setspace} % Optional: for nicer spacing
\usepackage{parskip} % Optional: spacing between paragraphs without indentation
\usepackage[margin=1in]{geometry}

\usepackage{fancyhdr}
\pagestyle{fancy}
\fancyhf{}

\usepackage{amsmath}
\usepackage{listings}

\begin{document}
\thispagestyle{empty}

\begin{center}
    \textbf{Львівський національний університет імені Івана Франка}\\[1ex]
    Кафедра дискретного аналізу та інтелектуальних систем
\end{center}

\vspace{5cm} % Space before main title

\begin{center}
    \Large\textbf{Курс «Моделі подання знань»}\\[4ex]
    \huge\textbf{ЗВІТ}\\[3ex]
    \Large з семінарських занять за матеріалами лекцій 9 - 11 (пакет завдань № 4)
\end{center}

\vspace{2cm}
\begin{flushright}
    Виконав:\\
    Студент групи ПМіМ-12\\
    Паньків Олесь
\end{flushright}

\fancyfoot{Львів 2025}

\newpage
\pagestyle{plain}

\textbf{Завдання 3.} Припустимо, що база знань містить тільки одне висловлення
$\exists x\; AsHighAs(x, Everest)$. Які з наведених нижче фактів є правильними
результатами застосування правила конкретизації з квантором існування?

\begin{itemize}
    \item[а)] $AsHighAs(Everest, Everest).$
    \item[б)] $AsHighAs(Kilimanjaro, Everest).$
    \item[в)] $AsHighAs(Kilimanjaro, Everest) \land AsHighAs(BenNevis, Everest)$ \\
    \hspace*{1.8em}(після двох застосувань).
\end{itemize}

\textbf{Розв'язок}:

У варіанті а)
\[
    AsHighAs(Everest,Everest)
\]

як $x$ використано вже існуючу константу - $Everest$. Тому варіант
неправельний.

Інші висновки б) та в) коректні оскільки символи $Kilimanjaro$ та $BenNevis$ не
були у базі даних.

\textbf{Отже, відповідь}: б, в

\vspace{1cm}
\textbf{Завдання 6(г).} Ці питання стосуються питань підстановки та сколемізації.
\begin{itemize}
    \item[г)] Поширена помилка серед студентів полягає в тому, що при уніфікації
              вони часто виконують підстановку терму замість константи Сколема,
              а не підстановку терму замість змінної. Наприклад, вони говорять,
              що формули $P(Sk1)$ і $P(A)$ можна уніфікувати за допомогою
              підстановки ${Sk1/A}$. Наведіть приклад, коли це призводить до
              неправильного висновку.
\end{itemize}

\textbf{Розв'язок}:

Маємо висловлення $\exists x\; P(x)$, яке після сколемізації набуває вигляду
$P(Sk1)$, де $Sk1$ — сколем-константа.

Типовою помилкою є виконання підстановки $\{Sk1 / A\}$ з метою уніфікації
$P(Sk1)$ та $P(A)$. Така підстановка передбачає, що $Sk1 = A$, тобто ми знаємо,
що саме об'єкт $A$ задовольняє $P$. Однак це припущення необґрунтоване:
сколем-константа означає лише факт існування, а не ідентичність з будь-яким
конкретним термом. Тому така підстановка порушує логіку сколемізації.

\textit{Приклад.}
У базі знань маємо
\[
\exists x\; P(x)
\]
\[
\neg P(A)
\]

Студент помилково уніфікує $P(Sk1)$ з $P(A)$ підстановкою $\{Sk1 / A\}$,
і отримує протиріччя:
\[
P(A) \quad \text{і} \quad \neg P(A)
\]

Але оскільки ми не маємо підстав вважати, що $Sk1 = A$, це протиріччя
не є логічно обґрунтованим.

\vspace{1cm}
\textbf{Завдання 13(б).} Сформулюйте на мові Prolog перелічені нижче питання
щодо відношення parent, яке описано в підрозділі «Визначення відношень на снові
фактів» у лекції 10 (див. рис. 22).

\begin{itemize}
    \item[б)] Чи має Ліз дитину?
\end{itemize}

\textbf{Розв'язок}:
\begin{lstlisting}[language=Prolog]
?- parent(liz, _).
\end{lstlisting}

\end{document}
