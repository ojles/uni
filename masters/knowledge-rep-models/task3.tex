\documentclass[a4paper,14pt]{extarticle} % <<< use extarticle for 14pt size

\usepackage[T2A]{fontenc}
\usepackage[utf8]{inputenc}
\usepackage[ukrainian]{babel}
\usepackage{setspace} % Optional: for nicer spacing
\usepackage{parskip} % Optional: spacing between paragraphs without indentation
\usepackage[margin=1in]{geometry}

\usepackage{fancyhdr}
\pagestyle{fancy}
\fancyhf{}

\usepackage{amsmath}

\begin{document}
\thispagestyle{empty}

\begin{center}
    \textbf{Львівський національний університет імені Івана Франка}\\[1ex]
    Кафедра дискретного аналізу та інтелектуальних систем
\end{center}

\vspace{5cm} % Space before main title

\begin{center}
    \Large\textbf{Курс «Моделі подання знань»}\\[4ex]
    \huge\textbf{ЗВІТ}\\[3ex]
    \Large з семінарських занять за матеріалами лекцій 7 та 8 (пакет завдань № 3)
\end{center}

\vspace{2cm}
\begin{flushright}
    Виконав:\\
    Студент групи ПМіМ-12\\
    Паньків Олесь
\end{flushright}

\fancyfoot{Львів 2025}

\newpage
\pagestyle{plain}

\textbf{Завдання 7.} Рівняння (8.1) у лекції 8 визначає умову, за якої квадрат вітряний. Тут ми
розглянемо два інших шляхи описати цей аспект світу вампуса.\\
\textbf{б)} Ми можемо записати причинно-наслідкові правила, що ведуть від причини до наслідку.
Одне очевидне причинно-наслідкове правило полягає в тому, що яма спричиняє, що всі
сусідні квадрати будуть вітряними. Запишіть це правило в логіці першого порядку, поясніть,
чому воно неповне порівняно з рівнянням (8.1), і додайте відсутню аксіому.

\textbf{Розв'язок}: Запишемо причинно-наслідкове правило, що яма спричиняє
сусідні вітряні квадрати:

\[
    \forall s\; \text{Pit}(s) \Rightarrow \left( \forall r\; \text{Adjacent}(r, s) \Rightarrow \text{Breezy}(r) \right)
\]

Ця логіка не забороняє існування вітру без наявності ями в сусідніх клітинках:

\[
    \exists s\; \left( \left( \forall r\; \text{Adjacent}(r, s) \Rightarrow \neg\text{Pit}(r) \right) \land \text{Breezy}(s) \right)
\]

Це стає проблемою тоді, коли агент відчуває вітер і намагається зробити висновок про наявність ями.
Без додаткового обмеження агент не може бути впевненим, що вітер спричинений ямою, бо формально
допускається, що вітер міг виникнути сам по собі. Запишемо аксіому щоб виправити це:

\[
    \forall s\; \left( \forall r\; \text{Adjacent}(r, s) \Rightarrow \neg\text{Pit}(r) \right) \Rightarrow \neg\text{Breezy}(s)
\]


Для повного та коректного опису взаємозв'язку між ямами та вітром необхідно використовувати
обидві формули: перша встановлює, що наявність ями спричиняє вітер у суміжних клітинках, а
друга — гарантує відсутність вітру за відсутності суміжних ям.

\vspace{1cm} % Space before main title
\textbf{Завдання 10.} Використовуючи предикати \textit{Parent(p, q)}, \textit{Female(p)}
і константи \textit{Joan}, \textit{Kevin} з очевидними значеннями, висловіть кожне з поданих речень
висловленням у логіці першого порядку. (Використайте $\exists!$ для квантора «існує точно один».)

ґ) Джоана має принаймні одну дитину з Кевіним, і не має дітей з будь-ким іншим.

\indent \textbf{Розв'язок}: Існує хоча б одна дитина, яка є спільною для Joan та Kevin.
Це можна виразити предикатами
Parent(Joan, c) та Parent(Kevin, c) для деякого c (c - 'child', тобто дитина).

\[
    \exists c\; \, \text{Parent}(\text{Joan}, c) \land \text{Parent}(\text{Kevin}, c)
\]

Далі потрібно забезпечити, що у Joan немає дітей з іншими особами. Для цього перевіряємо:
для будь-якої дитини d, яка є дитиною Joan, будь-який її батько p повинен бути або Joan, або Kevin.
Таким чином, ми забороняємо існування інших батьків, окрім Joan та Kevin.

Тому остаточна формула має вигляд:

\[
    \begin{aligned}
        \exists c\; & \, \text{Parent}(\text{Joan}, c) \land \text{Parent}(\text{Kevin}, c) \land \\
                    & \forall d, p\; \left( \text{Parent}(\text{Joan}, d) \land \text{Parent}(p, d) \right)
                    & \Rightarrow (p = \text{Joan} \lor p = \text{Kevin})
    \end{aligned}
\]

\end{document}
