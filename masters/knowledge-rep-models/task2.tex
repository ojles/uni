\documentclass[a4paper,14pt]{extarticle} % <<< use extarticle for 14pt size

\usepackage[T2A]{fontenc}
\usepackage[utf8]{inputenc}
\usepackage[ukrainian]{babel}
\usepackage{setspace} % Optional: for nicer spacing
\usepackage{parskip} % Optional: spacing between paragraphs without indentation
\usepackage[margin=1in]{geometry}
\usepackage{graphicx}
\usepackage{array}
\usepackage{enumitem}

\usepackage{fancyhdr}
\pagestyle{fancy}
\fancyhf{}

\usepackage{amsmath}

\begin{document}
\thispagestyle{empty}

\begin{center}
    \textbf{Львівський національний університет імені Івана Франка}\\[1ex]
    Кафедра дискретного аналізу та інтелектуальних систем
\end{center}

\vspace{5cm} % Space before main title

\begin{center}
    \Large\textbf{Курс «Моделі подання знань»}\\[4ex]
    \huge\textbf{ЗВІТ}\\[3ex]
    \Large з семінарських занять за матеріалами лекцій 5 та 6 (пакет завдань № 2)
\end{center}

\vspace{2cm}
\begin{flushright}
    Виконав:\\
    Студент групи ПМіМ-12\\
    Паньків Олесь
\end{flushright}

\fancyfoot{Львів 2025}

\newpage
\pagestyle{plain}

\textbf{Завдання 16.}
\begin{enumerate}[label=]
    \item а) Поясніть, чому будь-яке висловлення у формі непорожнього диз’юнкта
        само по собі є виконуваним.
    \item в) Висловлення задано у 3-CNF формі, причому в кожному диз’юнкті є саме
        три різні змінні. Якою має бути найменша кількість диз’юнктів, щоб це
        висловлення було невиконуваним? Побудуйте таке висловлення.
\end{enumerate}

\textbf{Розв'язок}:

\textbf{а)} Непорожній диз’юнкт має принаймні один літерал.

Будь-який диз’юнкт виконується тоді, коли хоча б один літерал у ньому істинний.

Можна знайти набір значень змінних, де цей літерал істинний, тому такий диз’юнкт
завжди має модель.

Отже, будь-яке висловлення у формі одного непорожнього диз’юнкта – виконуване.

\textbf{в)} Щоб зробити 3-CNF формулу невиконуваною, коли в кожному диз’юнкті по
три різні змінні, потрібні щонайменше 8 диз’юнктів.

\textbf{Приклад невиконуваного висловлення:}
\[
\begin{aligned}
& (x_1 \lor x_2 \lor x_3) \\
& (\neg x_1 \lor x_2 \lor x_3) \\
& (x_1 \lor \neg x_2 \lor x_3) \\
& (x_1 \lor x_2 \lor \neg x_3) \\
& (\neg x_1 \lor \neg x_2 \lor x_3) \\
& (\neg x_1 \lor x_2 \lor \neg x_3) \\
& (x_1 \lor \neg x_2 \lor \neg x_3) \\
& (\neg x_1 \lor \neg x_2 \lor \neg x_3)
\end{aligned}
\]

Ці 8 диз’юнктів покривають усі можливі набори значень трьох змінних. Тому немає жодної моделі, що задовольняє всі диз’юнкти водночас.

\vspace{1cm} % Space before main title
\textbf{Завдання 21.} Використовуючи метод Девіса і Патнема, довести,
що наступні висловлення виконувані.

\begin{enumerate}[label=]
    \item б) $(P \lor Q) \land (\neg P \lor Q) \land R. $
\end{enumerate}

\textbf{Розв'язок}:

\textbf{б)}
\begin{itemize}
  \item З останньої частини отримаємо: \( R = 1 \).
  \item Перша частина: \((P \lor Q) = 1\). Для цього достатньо, щоб \( P = 1 \) або \( Q = 1 \).
  \item Друга частина: \((\neg P \lor Q) = 1\). Це можливо, якщо:
    \begin{itemize}
      \item Якщо \( P=1 \), то мусить бути \( Q=1 \).
      \item Якщо \( P=0 \), тоді байдуже, бо \(\neg P=1\).
    \end{itemize}
  \item Один приклад істинного набору:  
  \[
  P=1, \quad Q=1, \quad R=1
  \]
  \item Перевірка:  
  \[
  (1 \lor 1)=1, \quad (\neg 1 \lor 1)=(0 \lor 1)=1, \quad R=1
  \]
\end{itemize}

Отже, формула має модель – висловлення виконуване.

\vspace{1cm} % Space before main title
\textbf{Завдання 23.} Скільки часу потрібно (пропорційно до $\supset$), щоб довести
$KB \models \alpha$ за допомогою алгоритму DPLL, якщо $\alpha$ – літерал, який уже є
в базі знань $KB$. Поясніть отримані результати.

\textbf{Розв'язок}: Час виконання алгоритму буде пропорційний \textbf{кількості символів}
та \textbf{кількості одиничних диз’юнктів}.  

Алгоритм DPLL працює наступним чином:
\begin{enumerate}
  \item Спочатку він видаляє всі чисті символи.
  \item Потім обробляє одиничні диз'юнкти, поки не зустріне або $\alpha$, або $\neg \alpha$.
  \item Побачивши, що і $\alpha$, і $\neg \alpha$ не можуть бути істинними одночасно,
      алгоритм одразу визначає, що припущення про хибність $\alpha$ призводить до протиріччя.
  \item Отже, $\alpha$ є істинним.
\end{enumerate}

\end{document}
